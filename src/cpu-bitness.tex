%%%%%%%%%%%%%%%%%%%%%%%%%%%%%%%%%%%%%%%%%
% Journal Article
% LaTeX Template
% Version 1.4 (15/5/16)
%
% This template has been downloaded from:
% http://www.LaTeXTemplates.com
%
% Original author:
% Frits Wenneker (http://www.howtotex.com) with extensive modifications by
% Vel (vel@LaTeXTemplates.com)
%
% License:
% CC BY-NC-SA 3.0 (http://creativecommons.org/licenses/by-nc-sa/3.0/)
%
%%%%%%%%%%%%%%%%%%%%%%%%%%%%%%%%%%%%%%%%%

%----------------------------------------------------------------------------------------
%	PACKAGES AND OTHER DOCUMENT CONFIGURATIONS
%----------------------------------------------------------------------------------------

\documentclass[twoside,twocolumn]{article}

\usepackage[sc]{mathpazo} % Use the Palatino font
\usepackage[T1]{fontenc} % Use 8-bit encoding that has 256 glyphs
\linespread{1.05} % Line spacing - Palatino needs more space between lines
\usepackage{microtype} % Slightly tweak font spacing for aesthetics

\usepackage[english]{babel} % Language hyphenation and typographical rules

\usepackage[hmarginratio=1:1,top=32mm,columnsep=20pt]{geometry} % Document margins
\usepackage[hang, small,labelfont=bf,up,textfont=it,up]{caption} % Custom captions under/above floats in tables or figures
\usepackage{booktabs} % Horizontal rules in tables

\usepackage{lettrine} % The lettrine is the first enlarged letter at the beginning of the text

\usepackage[shortlabels]{enumitem} % Customized lists
\setlist[itemize]{noitemsep} % Make itemize lists more compact
\setlist[enumerate, 1]{1\textsuperscript{o}} % Numberd lists with right parenthesis

\usepackage{abstract} % Allows abstract customization
\renewcommand{\abstractnamefont}{\normalfont\bfseries} % Set the "Abstract" text to bold
\renewcommand{\abstracttextfont}{\normalfont\small\itshape} % Set the abstract itself to small italic text

\usepackage{titlesec} % Allows customization of titles
\renewcommand\thesection{\Roman{section}} % Roman numerals for the sections
\renewcommand\thesubsection{\roman{subsection}} % Roman numerals for subsections
\titleformat{\section}[block]{\large\scshape\centering}{\thesection.}{1em}{} % Change the look of the section titles
\titleformat{\subsection}[block]{\large}{\thesubsection.}{1em}{} % Change the look of the section titles

\usepackage{fancyhdr} % Headers and footers
\pagestyle{fancy} % All pages have headers and footers
\fancyhead{} % Blank out the default header
\fancyfoot{} % Blank out the default footer
\fancyhead[C]{CPU Bitness} % Custom header text
\fancyfoot[RO,LE]{\thepage} % Custom footer text

\usepackage{titling} % Customizing the title section

\usepackage{hyperref} % For hyperlinks in the PDF

\usepackage{tablefootnote} % For using footnotes in tables

%----------------------------------------------------------------------------------------
%	TITLE SECTION
%----------------------------------------------------------------------------------------

\setlength{\droptitle}{-4\baselineskip} % Move the title up

\pretitle{\begin{center}\Huge\bfseries} % Article title formatting
\posttitle{\end{center}} % Article title closing formatting
\title{CPU Bitness} % Article title
\author{%
\textsc{Simon Rupf} \\[1ex]
\normalsize \href{mailto:simon@rupf.net}{simon@rupf.net}
}
\date{\today} % Leave empty to omit a date
\renewcommand{\maketitlehookd}{%
\begin{abstract}
\noindent This article collects key properties of different CPU types, from a low level
programming point of view. At the core of this document, this data is presented in a
single, concise table, allowing a nuanced comparison of these different CPU families.
\end{abstract}
}

%----------------------------------------------------------------------------------------

\begin{document}

% Print the title
\maketitle

%----------------------------------------------------------------------------------------
%	ARTICLE CONTENTS
%----------------------------------------------------------------------------------------

\section{Introduction}

\lettrine[nindent=0em,lines=3]{W}hen discussing properties of a computers processing
power, one particular property often used to classify a system is it's central processing
unit's (CPU) bitness. A CPU is said to be, for example, of the 8-bit type and therefore
compared to other systems sporting CPUs of the same class. When looking closer at such a
CPU, the situation is, of course, more complex. This article tries to collect those
details over a broad range of CPUs, allowing for a more nuanced comparison of these CPUs.
It also attempts to highlight the somewhat arbitrary nature of these CPU bitness
categorizations.

%------------------------------------------------

\begin{table*}
\centering % Center table
\begin{tabular}{lrrrrrrl}
&& Count & \multicolumn{4}{c}{Width in Bits} \\
\cmidrule(r){3-7}
CPU & Year & \multicolumn{2}{c}{Instructions} & Data & Address & Bus & Used in \\
\toprule
AGC\tablefootnote{CPU isn't implemented as an integrated circuit, it uses words of 15
bits with a 1 bit parity, instructions are one word, when the instruction EXTEND is
used, the next instruction is decoded using a different code set, hence these extended
instructions are 2 words long instead of one, the only double precision fractional
numbers are supported as 2 word data type, all other data are 1 word, erasable memory
addresses need 8 bits, fixed memory addresses require 16} \cite{agc} & 1961 & 34/97\tablefootnote{34 instructions were implemented in the CPU hardware, but most programs were written in an interpreted instruction set of 97 higher-level instructions} & 15/30 & 15/30 & 8/16 & N/A & Apollo Spacecraft \\
\hline
Intel 4004\tablefootnote{bus has one pin to select ROM and 4 pins to select the RAM bank,
4 pins are used bi-directional for both address selection and data transfers} \cite{intel4004} & 1971 & 46 & 8/16 & 4 & 12 & 9 & Busicom 141-PF \\
\hline
Intel 8008\tablefootnote{bus is used bi-directional for both address selection and data
transfers} \cite{intel8008} & 1972 & 48 & 8/16/22 & 8 & 14 & 8 & MCS-8, Mark-8 \\
\hline
Intel 8080\tablefootnote{separate address and data bus} \cite{intel8080} & 1974 & 244 & 8/16/24 & 8 & 16 & 24 & MITS Altair 8800 \\
\hline
Motorola 6800 \cite{mc6800} & 1974 & 72 & 8/16/24 & 8 & 16 & 24 & Sphere \\
\hline
MOS 6502\tablefootnote{designed to be an improved and low-cost evolution on the Motorola
6800, MOS 6501 was even pin compatible, but the instruction set that is incompatible to
the 6800} \cite{mos6500} & 1975 & 56 & 8/16/24 & 8 & 16 & 24 & Apple I \\
\hline
Intel 8085\tablefootnote{lower 8 bits are used bi-directionally for data transfer, same
instruction set as the 8080} \cite{intel8085} & 1976 & 246 & 8/16/24 & 8 & 16 & 16 & TRS-80 Model 100 \\
\hline
Zilog Z80\tablefootnote{designed to be binary compatible with the Intel 8080, but
offering additional instructions} \cite{z8400} & 1976 & 696 & 8/16/24/32 & 8/16 & 16 & 24 & Sinclair ZX80 \\
\hline
Intel 8086\tablefootnote{lower 16 bits are used bi-directionally for data transfer,
highest 4 bits are used for signals and segment selection} \cite{intel8086} & 1978 && 8/16/24/32 & 8/16 & 20 & 20 & Compaq Deskpro \\
\hline
Intel 8088\tablefootnote{data is fetched 8 bits at a time, but internally stored in 16
bit registers, lowest 8 bits are used bi-directionally for data transfer, highest 4
bits are used for signals and segment selection} \cite{intel8088} & 1979 && 8/16/24/32 & 8/16 & 20 & 20 & IBM PC 5150 \\
\hline
Motorola 6809 \cite{mc6809} & 1979 & 72 & 8/16/24/32 & 8/16 & 16 & 24 & TRS-80 CoCo \\
\hline
MOS 6510\tablefootnote{differs in pin out to the MOS 6502, adding 8 lanes for general
purpose I/O to the bus, only used in the Commodore 64 which got introduced in 1982} \cite{mos6510} & 1982 & 56 & 8/16/24 & 8 & 16 & 32 & Commodore 64 \\
\hline
Hitachi 64180\tablefootnote{designed to be binary compatible with the Zilog Z80, offers
12 additional instructions, among them multiplication, option to address 512 KiB (19
address lines) or 1 MiB (20 address lines) of memory, got licensed to Zilog which brands
it Z180} \cite{hd64180} & 1985 & 708 & 8/16/24/32 & 8/16 & 20 & 28 & Victor HC-90 \\
\end{tabular}
\caption{CPU Bitness}
\label{tab:bitness}
\end{table*}

\begin{table}
\centering % Center table
\begin{tabular}{ll}
Vendor & Reference \\
\toprule
AGC & \cite{agc} \\
\hline
Hitachi & \cite{hd64180} \\
\hline
Intel & \cite{intelquick} \\
\hline
MOS & \cite{ieee_computer_mos6502} \\
\hline
Motorola & \cite{motorolamanual} \\
\hline
Zilog & \cite{z80dev} \\
\end{tabular}
\caption{Introduction year reference}
\label{tab:introduction}
\end{table}

%------------------------------------------------

\section{Methods}

As much as possible, the bit widths were looked up in data sheets or product
specifications, as they present the CPUs features in a compact manner. As some CPUs have
been manufactured over long periods of time and later revisions feature improved designs,
this article tries to source the numbers from sheets published as early as possible in
the products life cycle.

Sourcing early information on the Zilog Z80 CPU proved to be a particular challenge, as
some revisions of this chip are still produced by the same company at the time this
article is written. This design has been in production since 1976 and a lot of material
has been published for it over the years.

After comparing several CPU models, the following properties were selected for comparison:

\begin{itemize}
\item Instruction count
\item Instruction width
\item Data width
\item Address width
\item Bus width
\end{itemize}

\subsection{Instruction count}

While proof-of-concept CPU architectures exist that work with a single, universal
instruction, all CPUs used for practical applications will provide a number of different
instructions. These will usually include general purpuse instructions to add and
subtract numbers stored in CPU-internal memory (called registers), store and load data
from CPU-external memory or doing other input and output operations (I/O), conditional
branching and to jump to a different location. Often additional mathematical operations
are supported or operations that combine conditionals and jumps. CPUs tend to be labeled
as having been designed with more complex or reduced instruction set computing
(CISC/RISC) in mind. Early CPUs tended to implement less instructions purely due to
manufacturing constraints. Later CPUs tend to offer a very large instruction set while
internally working with a much smaller instruction set and using an interpreter to
convert one set into the other, which may itself be loaded from a re-programmable memory
device, allowing the CPU to gain new instructions during it's lifetime via an update of
its firmware.

For the purposes of this document, the instruction count of the earliest data sheet found
are listed. The counts do include all variant instructions, if they use different binary
representations, but omits duplicate or undocumented instructions. An example from the
Intel 8080 are several undocumented instructions that all act as a No-Op that are not
counted, while the MOV instruction has two arguments for selecting destination and source
of the move, adding a total of 63 combinations (7 registers plus memory as either source
or destination, excluding the illegal memory to memory combination, which would require
two address arguments).

\subsection{Instruction width}

As described first by John Von Neumann \cite{edvac}, instructions and data for the CPU
can both be stored in a single storage. In order to address both using the same
mechanism, these need be stored in same increments. The instruction width as presented
in this article records the different lengths possible for an instruction, which can
consist of an instruction code and optionally additional parameters representing data to
operate on and/or an address to a memory location.

\subsection{Data width}

In order to handle different types of data efficiently, CPUs may offer instructions that
manipulate these directly. Some data types might require more bits then fit in a single
address. For these, the CPU will need to handle a multiple of the smallest data width,
which is called a word.

\subsection{Address width}

In order for the CPU to address every word of data in it's memory they all need to be
addressable. The address width limits the maximum amount of memory that can be accessed.
Some CPUs implement banking or segmentation of memory in order to switch between
different parts of memory that would otherwise be outside of the address range. In such
cases, the width of the address and the additional bits of the bank or segment are
summed up.

\subsection{Bus width}

A CPU can have one or multiple buses to transmit and receive addresses, data and signals
with the rest of the system. For the purpose of this article only the number of lanes for
data and addresses were considered.

%------------------------------------------------

\section{Results}

To avoid repetition, the references on the years of introduction of the CPUs got
collected in Table \ref{tab:introduction}. The main Table \ref{tab:bitness} collects
information on the bit width of key aspects of the CPUs.

%------------------------------------------------

\section{Noteworthy Cases}

One might assume that an ideal CPU design would use the same amount of bits for address
space as for it's data, in order to simplify the design as much as possible. Reviewing
Table \ref{tab:bitness}, two points should stand out: 1) the instruction width is a
multiple of the data width and 2) none of the listed CPUs use an address width matching
their data width. These observations can be explained as follows:

\begin{enumerate}[1)]
\item Instructions will be at least as long as the data, as per the Von Neumann
architecture \cite{edvac}. But they might act on a piece of data or an address.
\item For the majority of CPU designs, the amount of available memory exceeds what could
be addressed within the width of the data. For example, a CPU operating on 8-bit data
will want to address more then $2^8 = 256$ addresses.
\end{enumerate}

The sections below detail some CPU models that took particularly noteworthy trade-offs.

\subsection{Apollo Guidance Computer}

The information on this device is based on secondary literature \cite{agc}. This system
predates the availability of integrated circuits (IC) and so it's central processing unit
got implemented using discrete electronic components. This makes it difficult to discern
a data or address bus easily, although it does clearly separate two types of memory,
erasable core memory and fixed core rope storage. Both get addressed independently and
use different banking and address widths. The erasable memory servers both for what would
now be IC-internal CPU registers, as well as for storing data or in flight updated
programs. The two types of storage are not mapped into a single address space and
different instruction are used to access data or instructions in each. Instructions can
directly be executed from both types of storage and special instructions are used to
transfer control between the two. Both types of storage can hold data and instructions,
so it is still a Von Neumann architecture \cite{edvac}, even though the system doesn't
make use of a unified address space as many later designs do, were RAM and ROM get mapped
into a single address space, making transfer between the two seamless.

It is debatable if this system can be classified as 15 or 16 bit system. Data can only be
stored as 15 bits, or 30 bits, in the case of double precision fractionals. The 16th bit
of each word serves as parity and in a few cases as an overflow flag, but in most cases
can't be accessed by the software - the overflow handling requires extra operations and
so that bit can be indirectly accessed. It is therefore also difficult give an amount of
data storage in bits, since the amount of data storage capacity that is accessible by the
software differs from the physical amount of bits needed to represent it. And 8 bit bytes
are clearly incorrect or at least misleading to represent both bit widths. To avoid
confusion the term word is used when referring to a 15 bit piece of data. Considering
that modern ECC memory modules do not include memory used for parity information in their
capacity listings, this article chooses to list this device as a 15 bit architecture.

It is noteworthy that while the data widths are larger then subsequent integrated circuit
CPU designs, the address width for both memory types is smaller then the data width.
Storing addresses as data in memory is therefore rather wasteful. This is mitigated by
the structure of the instructions, of which there are two types: Those consisting of 3
bits of Opcode, followed by 2 bits of Quarter code, leaving 10 bits for the storage
address. Since the system uses banking to access both types of storage, this is
sufficient to store the address within the currently selected bank for both types of
memory. One of these instructions is the EXTEND instruction. When it is given, the
following word contains an instruction that gets decoded using a different instruction
set. This three staged design illustrates how the instruction architecture had evolved
over time.

\subsection{Intel 8080 \& it's Evolutions}

We do already find a small number instructions that operate on 16 bits in the Intel 8080.
Later chip designs used only a single 5 Volt power source, instead of the 8080's 5 Volts,
12 Volts and negative 5 Volts, and they also evolved the available instruction set. The
Intel 8085 added two I/O related instructions and the Zilog Z80 adds a number of new
instructions, including two additional 16 bit arithmetic operations. None of these designs
support multiplication in their instruction set. \cite{softwaresolutions}

While these CPUs already support some instructions that operate on 16 bits of data, they
are not regarded to be 16 bit designs, because both their data bus and instruction set
are clearly centered around 8 bits of data.

\subsection{Intel 8086 \& 8088}

The Intel 8088 was released a year after the Intel 8086. It offered the same instruction
set, but reads and stores data 8 bits at a time from or into memory, where as the 8086
uses a 16 bit data bus. The full bus of the 8088 is 20 bits wide, using the lower 16 bits
to select addresses in memory. In order to access more then 64 KiB, it uses memory
segmentation. The segments are selected by the uppermost 4 bits on the bus.

This lets it address the same amount of memory as the 8086. The downside is that every
memory operation requires two full bus cycles, regardless if 8 or 16 bits of data are
to be processed.

Using only 8 bits to exchange data reduced the complexity of the bus design, as only 8
lanes need to connect to every memory chip, while the remaining 12 bits are used to
select the active chip to read from or write to.

Hence it trades clock cycles for reduced bus complexity and therefore cheaper circuit
board manufacturing.

On the topic of "CPU Bitness", note that the title in the 8088 data sheet calls it an
"8-BIT HMOS MICROPROCESSOR"\cite{intel8088}, while the 8086 is titled a "16-BIT HMOS
MICROPROCESSOR"\cite{intel8086}. The first IBM Personal Computer, model 5150, choose
the Intel 8088 rather then the 8086, but is usually categorized as 16 bit.

\subsection{Motorola 6800 \& MOS 6500}

The Motorola 6800 did support 72 instructions and was designed to be easily adapted by
customers for a wide range of applications from industrial to aerospace and automotive
use. One of the engineers, Chuck Peddle, would accompany salespeople on customer visits
and witnessed these being put off by the high cost of the processor, which was initially
sold for USD 360. He and other engineers began working on a new design with less
supported instructions, that would result in a simpler and cheaper to manufacture,
although less capable, chip.

In 1974, after Peddle had received a letter from Motorola management, ordering him to
stop working on his improvements, Chuck Peddle, Bill Mensch, Rod Orgill, Harry Bawcom,
Ray Hirt, Terry Holdt and Wil Mathys, switched to MOS Technology. At MOS they concluded
their work, which resulted in the MOS 6500 family of CPUs that offered 56 instructions.
While the differing instruction set made it incompatible to the Motorola 6800, it did
use a compatible pin layout, so that the 6500 chips could reuse many of the Motorola 6800
support chips and even could be dropped into existing board designs for the 6800. And it
sold for as low as USD 5, in bulk purchases.

Motorola did sue MOS Technology in 1975 for patent infringement and misappropriation of
trade secrets. In 1976 MOS agreed to stop producing the 6501 chip, paid Motorola USD
200,000 and returned confidential Motorola documents. By that time Motorola had lowered
the price of the 6800 to USD 35.

The MOS 6510 is notable to having introduced dedicated pins for general purpose I/O on
the IC, although it only got used in the Commodore 64. Without such port-mapped I/O, the
address and data lanes have to serve double duty for memory-mapped I/O. Independently of
these is direct memory access (DMA), where I/O devices directly write and read from
memory, independent of the CPU. While DMA frees CPU cycles, it introduces complexity to a
system's design, as multiple devices have to access a shared bus, without interfering
with each other. Port-mapped I/O does allow a CPU more precise control over I/O signals,
which is important in real time computing or for signals that require precise timing,
without having to introduce dedicated I/O handling devices to the design.

%----------------------------------------------------------------------------------------
%	REFERENCE LIST
%----------------------------------------------------------------------------------------

\begin{thebibliography}{99} % Bibliography

\bibitem{edvac}
\textit{The First Draft Report on the EDVAC}.
Memory that stores data and instructions: Page 42, Section 15.6.
John Von Neumann, June 1945.

\bibitem{agc}
\textit{The Apollo Guidance Computer}.
Data: Pages 20 \& 27, Address: Pages 42-43, Instructions: Pages 71-73.
Frank O'Brien; Praxis Publishing Limited, 2010.
ISBN 978-1-4419-0876-6

\bibitem{hd64180}
\textit{HD64180 Hardware Manual} 4th Edition.
Bus: Page 7, Address: Page 40-46, Data \& Instructions: Page 162-171, Year of Introduction: Page 209.
Hitachi Limited, 1989.

\bibitem{ieee_computer_mos6502}
\textit{MOS 6502 The Second of a Low Cost High Performance Microprocessor Family}.
IEEE Computer, Volume 8, Number 9, Page 38-39, September 1975.
\href{https://ieeexplore.ieee.org/stamp/stamp.jsp?tp=\&arnumber=1649550\&isnumber=34590}{ieeexplore.ieee.org/\ldots}

\bibitem{intelquick}
\textit{Intel Microprocessor Quick Reference Guide - Year}.
Intel Corporation, last modified: Tue, 12 Feb 2013 23:56:59 GMT.
\href{https://www.intel.com/pressroom/kits/quickrefyr.htm}{intel.com/pressroom/kits/quickrefyr.htm}

\bibitem{intelchips}
\textit{Intel Chips timeline}.
Intel Corporation, 2012.

\bibitem{intel4004}
\textit{Intel 4004 SINGLE CHIP 4-BIT P-CHANNEL MICROPROCESSOR}.
Address \& Bus: Page 2, Instructions \& Data: Page 3.
Intel Corporation, March 1987.

\bibitem{intel8008}
\textit{Intel 8008/8008-1 EIGHT-BIT MICROPROCESSOR}.
Address: Page 1, Bus: Page 2, Instructions \& Data: Page 3.
Intel Corporation, 1978.

\bibitem{intel8080}
\textit{Intel 8080A/8080A-1/8080A-2 8-BIT N-CHANNEL MICROPROCESSOR}.
Data \& Address: Page 1, Bus: Page 2, Instructions: Page 8.
Intel Corporation, 1986.

\bibitem{intel8080users}
\textit{Intel Intel 8080 Microcomputer Systems Users's Manual}.
CHAPTER 2 THE 8080 CENTRAL PROCESSOR UNIT: Page 15-34.
Intel Corporation, September 1975.

\bibitem{intel8085}
\textit{Intel 8085AH/8085AH-2/8085AH-1 8-BIT HMOS MICROPROCESSOR}.
Data: Page 1, Address \& Bus: Page 2, Instructions: Page 18.
Intel Corporation, September 1987.

\bibitem{intel8086}
\textit{Intel 8086 16-BIT HMOS MICROPROCESSOR 8086/8086-2/8086-1}.
Data: Page 1, Address \& Bus: Page 2, Instructions: Page 26.
Intel Corporation, September 1990.

\bibitem{intel8088}
\textit{Intel 8088 8-BIT HMOS MICROPROCESSOR 8088/8088-2}.
Data: Page 1, Address \& Bus: Page 2, Instructions: Page 26.
Intel Corporation, August 1990.

\bibitem{mos6500}
\textit{MCS6500 MICROPROCESSORS}.
Data \& Address: Page 1, Instructions: Page 6, Bus: Page 9.
MOS Technology Incorporated, May 1976.

\bibitem{mos6510}
\textit{6510 MICROPROCESSOR WITH I/O}.
Data, Address \& Bus: Page 1, Instructions: Page 9.
Commodore Semiconductor Group, ca. 1982-1985.

\bibitem{mc6800}
\textit{MC6800 MOS MICROPROCESSOR}.
Data, Address \& Bus: Page 1, Instructions: Page 13.
Motorola Incorporated, 1984.

\bibitem{mc6809}
\textit{MC6809 HMOS 8-BIT MICROPROCESSING UNIT}.
Data, Address \& Bus: Page 1, Instructions: Page 29.
Motorola Incorporated, 1983.

\bibitem{motorolamanual}
\textit{MOTOROLA MICROPROCESSORS DATA MANUAL}.
FIGURE 1-1: Page 13.
Motorola Incorporated, 1981.

\bibitem{z8400}
\textit{Zilog Product Specification Z8400/Z84C00 NMOS/CMOS Z80\textsuperscript{\textregistered} CPU Central Processing Unit}.
Data, Address \& Bus: Page 1, Instructions: Page 5.
Zilog Incorporated, ca. 1990.

\bibitem{z80dev}
\textit{Zilog Z80 Development System Product Specification}.
Zilog Incorporated, 1976.

\bibitem{softwaresolutions}
\textit{Software Solutions for Engineers And Scientists}.
Page 65: "The 8-bit microprocessors that preceded the 80x86 family (such as the Intel 8080, the Zilog Z80, and the Motorola) did not include multiplication."
Sanchez, Julio; Canton, Maria P.; Taylor \& Francis, 2008.
ISBN 978-1-4200-4302-0


\end{thebibliography}

%----------------------------------------------------------------------------------------

\end{document}
