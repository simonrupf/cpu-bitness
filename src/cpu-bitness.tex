%%%%%%%%%%%%%%%%%%%%%%%%%%%%%%%%%%%%%%%%%
% Journal Article
% LaTeX Template
% Version 1.4 (15/5/16)
%
% This template has been downloaded from:
% http://www.LaTeXTemplates.com
%
% Original author:
% Frits Wenneker (http://www.howtotex.com) with extensive modifications by
% Vel (vel@LaTeXTemplates.com)
%
% License:
% CC BY-NC-SA 3.0 (http://creativecommons.org/licenses/by-nc-sa/3.0/)
%
%%%%%%%%%%%%%%%%%%%%%%%%%%%%%%%%%%%%%%%%%

%----------------------------------------------------------------------------------------
%	PACKAGES AND OTHER DOCUMENT CONFIGURATIONS
%----------------------------------------------------------------------------------------

\documentclass[twoside,twocolumn]{article}

\usepackage[sc]{mathpazo} % Use the Palatino font
\usepackage[T1]{fontenc} % Use 8-bit encoding that has 256 glyphs
\linespread{1.05} % Line spacing - Palatino needs more space between lines
\usepackage{microtype} % Slightly tweak font spacing for aesthetics

\usepackage[english]{babel} % Language hyphenation and typographical rules

\usepackage[hmarginratio=1:1,top=32mm,columnsep=20pt]{geometry} % Document margins
\usepackage[hang, small,labelfont=bf,up,textfont=it,up]{caption} % Custom captions under/above floats in tables or figures
\usepackage{booktabs} % Horizontal rules in tables

\usepackage{lettrine} % The lettrine is the first enlarged letter at the beginning of the text

\usepackage[shortlabels]{enumitem} % Customized lists
\setlist[itemize]{noitemsep} % Make itemize lists more compact
\setlist[enumerate, 1]{1\textsuperscript{o}} % Numberd lists with right parenthesis

\usepackage{abstract} % Allows abstract customization
\renewcommand{\abstractnamefont}{\normalfont\bfseries} % Set the "Abstract" text to bold
\renewcommand{\abstracttextfont}{\normalfont\small\itshape} % Set the abstract itself to small italic text

\usepackage{titlesec} % Allows customization of titles
\renewcommand\thesection{\Roman{section}} % Roman numerals for the sections
\renewcommand\thesubsection{\roman{subsection}} % roman numerals for subsections
\titleformat{\section}[block]{\large\scshape\centering}{\thesection.}{1em}{} % Change the look of the section titles
\titleformat{\subsection}[block]{\large}{\thesubsection.}{1em}{} % Change the look of the section titles

\usepackage{fancyhdr} % Headers and footers
\pagestyle{fancy} % All pages have headers and footers
\fancyhead{} % Blank out the default header
\fancyfoot{} % Blank out the default footer
\fancyhead[C]{CPU Bitness} % Custom header text
\fancyfoot[RO,LE]{\thepage} % Custom footer text

\usepackage{titling} % Customizing the title section

\usepackage{hyperref} % For hyperlinks in the PDF

\usepackage{tablefootnote} % For using footnotes in tables
\makeatletter
\newcommand{\spewfootnotes}{%
\tfn@tablefootnoteprintout%
\global\let\tfn@tablefootnoteprintout\relax%
\gdef\tfn@fnt{0}%
}
\makeatother

%----------------------------------------------------------------------------------------
%	TITLE SECTION
%----------------------------------------------------------------------------------------

\setlength{\droptitle}{-4\baselineskip} % Move the title up

\pretitle{\begin{center}\Huge\bfseries} % Article title formatting
\posttitle{\end{center}} % Article title closing formatting
\title{CPU Bitness} % Article title
\author{%
\textsc{Simon Rupf} \\[1ex]
\normalsize \href{mailto:simon@rupf.net}{simon@rupf.net}
}
\date{\today} % Leave empty to omit a date
\renewcommand{\maketitlehookd}{%
\begin{abstract}
\noindent When discussing properties of a computers processing power, one particular
property often used to classify a system is it's central processing unit's (CPU) bitness.
A CPU is said to be, for example, of the 8-Bit type and therefore compared to other
systems sporting CPUs of the same class. When looking at such a CPU more in-depth the
situation is, of course, more complex. This article tries to collect those details over
a broad range of CPUs, allowing for a more nuanced comparison of these CPUs. It also
attempts to highlight the somewhat arbitrary nature of these CPU Bitness categorizations.
\end{abstract}
}

%----------------------------------------------------------------------------------------

\begin{document}

% Print the title
\maketitle

%----------------------------------------------------------------------------------------
%	ARTICLE CONTENTS
%----------------------------------------------------------------------------------------

\section{Introduction}

\lettrine[nindent=0em,lines=3]{L} orem ipsum dolor sit amet, consectetur adipiscing elit.

%------------------------------------------------

\section{Methods}

Maecenas sed ultricies felis. Sed imperdiet dictum arcu a egestas. 
\begin{itemize}
\item Donec dolor arcu, rutrum id molestie in, viverra sed diam
\item Curabitur feugiat
\item turpis sed auctor facilisis
\item arcu eros accumsan lorem, at posuere mi diam sit amet tortor
\item Fusce fermentum, mi sit amet euismod rutrum
\item sem lorem molestie diam, iaculis aliquet sapien tortor non nisi
\item Pellentesque bibendum pretium aliquet
\end{itemize}

%------------------------------------------------

\section{Results}

\begin{table*}[h]
\centering % Center table
\begin{tabular}{lrrrrrl}
\toprule
&&\multicolumn{4}{c}{Width in Bits} \\
\cmidrule(r){3-6}
CPU & Year & Instructions & Data & Address & Bus & Used in \\
\toprule
Intel 4004 & 1971 \cite{intelquick} &  &  &  &  &  \\
\hline
Intel 8008\tablefootnote{bus is used bi-directional for both address selection and data
transfers} \cite{intel8008} & 1972 \cite{intelquick} & 8/16/22 & 8 & 14 & 8 & MCS-8, Mark-8 \\
\hline
Intel 8080\tablefootnote{separate address and data bus} \cite{intel8080} & 1974 \cite{intelquick} & 8/16/24 & 8 & 16 & 24 & MITS Altair 8800 \\
\hline
Intel 8085\tablefootnote{lower 8 bits are used bi-directionally for data transfer, same
instruction set as the 8080} \cite{intel8085} & 1976 \cite{intelquick} & 8/16/24 & 8 & 16 & 16 & TRS-80 Model 100 \\
\hline
Intel 8086\tablefootnote{lower 16 bits are used bi-directionally for data transfer,
highest 4 bits are used for signals and segment selection} \cite{intel8086} & 1978 \cite{intelquick} & 8/16/24/32 & 8/16 & 20 & 20 & Compaq Deskpro \\
\hline
Intel 8088\tablefootnote{data is fetched 8 bits at a time, but internally stored in 16
bit registers, lowest 8 bits are used bi-directionally for data transfer, highest 4
bits are used for signals and segment selection} \cite{intel8088} & 1979 \cite{intelquick} & 8/16/24/32 & 8/16 & 20 & 20 & IBM PC 5150 \\
\hline
Zilog Z80\tablefootnote{designed to be binary compatible with Intel 8080, but supporting
additional instructions} & 197x & 8-24 & 8/16 & 16 & 8 & Sinclair ZX Spectrum \\
\hline
MOS 6502\tablefootnote{designed to be an improved and low-cost evolution of the Motorola 6800 -
MOS 6501 is even pin compatible, but both, 6501 \& 6502, use an instruction set that is
incompatible to the 6800} & 197x & 8-24 & 8 & 16 & 8 & Apple II \\
\hline
MOS 6510 & 197x & 8-24 & 8 & 16 & 8 & Commodore 64 \\
\hline
Motorola 6800 & 197x & 8-24 & 8 & 16 & 8 & Sphere \\
\hline
Motorola 6809 & 197x & 8-24 & 8 & 16 & 8 & TRS-80 Color Computer \\
\bottomrule
\end{tabular}
\caption{CPU Bitness table} % Add 'table' caption
\label{tab:bitness}
\end{table*}
\spewfootnotes

%------------------------------------------------

\section{Noteworthy Cases}

One could assume that an ideal CPU design would use the same amount of bits for address
space as for the it's data in order to simplify the design as much as possible. Reviewing
Table \ref{tab:bitness}, two things should be noted: 1) the instruction width is a
multiple of the data width and 2) none of the listed CPUs use an address width matching
their data width. These observations can be explained as follows:

\begin{enumerate}[1)]
\item Instructions will be at least as long as the data, as per the Von Neumann
architecture. But they might act on a piece of data or an address.
\item For the majority of CPU designs, the amount of available memory exceeds what could
be addressed within the width of the data. For example, a CPU operating on 8-bit data
will want to address more then $2^8 = 256$ addresses.
\end{enumerate}

The secions below detail some CPU models that took particularly noteworthy trade-offs.

\subsection{Intel 8086 \& 8088}

The Intel 8088 was released a year after the Intel 8086. It offered the same instruction
set, but reads and stores data 8 bits at a time from or into memory, where as the 8086
uses a 16 bit data bus. The full bus of the 8088 is 20 bits wide, using the lower 16 bits
to select addresses in memory. In order to access more then 64kB, it uses memory
segmentation. The segments are selected by the uppermost 4 bits on the bus.

This lets it address the same amount of memory as the 8086. The downside is that every
memory operation requires two full bus cycles, regardless if 8 or 16 bits of data are
to be processed.

Using only 8 bits to exchange data reduced the complexity of the bus design, as only 8
lanes need to connect to every memory chip, while the remaining 12 bits are used to
select the active chip to read from or write to.

Hence it trades clock cycles for reduced bus complexity and therefore cheaper circuit
board manufactoring.

The first IBM Personal Computer, model 5150, choose the Intel 8088 rather then the 8086.

On the topic of "CPU Bitness", note that the title in the 8088 datasheet calls it an
"8-BIT HMOS MICROPROCESSOR"\cite{intel8088}, while the 8086 is titled a "16-BIT HMOS
MICROPROCESSOR"\cite{intel8086}. Still, the IBM PC 5150 is usually categorized as 16 bit.

\subsection{Motorola 6800 \& MOS 6501/6502}

\subsection{Texas Instruments TMS9900}

%----------------------------------------------------------------------------------------
%	REFERENCE LIST
%----------------------------------------------------------------------------------------

\begin{thebibliography}{99} % Bibliography

\bibitem{intelquick}
\textit{Intel Microprocessor Quick Reference Guide - Year}.
Intel Corporation, last modified: Tue, 12 Feb 2013 23:56:59 GMT.
\href{https://www.intel.com/pressroom/kits/quickrefyr.htm}{intel.com/pressroom/kits/quickrefyr.htm}

\bibitem{intelchips}
\textit{Intel Chips timeline}.
Intel Corporation, 2012.

\bibitem{intel8008}
\textit{Intel 8008/8008-1 EIGHT-BIT MICROPROCESSOR}.
Address: Page 1, Bus: Page 2, Instructions \& Data: Page 3.
Intel Corporation, 1978.

\bibitem{intel8080}
\textit{Intel 8080A/8080A-1/8080A-2 8-BIT N-CHANNEL MICROPROCESSOR}.
Data \& Address: Page 1, Bus: Page 2, Instructions: Page 8.
Intel Corporation, 1986.

\bibitem{intel8080users}
\textit{Intel Intel 8080 Microcomputer Systems Users's Manual}.
CHAPTER 2 THE 8080 CENTRAL PROCESSOR UNIT, p.15-34.
Intel Corporation, September 1975.

\bibitem{intel8085}
\textit{Intel 8085AH/8085AH-2/8085AH-1 8-BIT HMOS MICROPROCESSOR}.
Data: Page 1, Address \& Bus: Page 2, Instructions: Page 18.
Intel Corporation, September 1987.

\bibitem{intel8086}
\textit{Intel 8086 16-BIT HMOS MICROPROCESSOR 8086/8086-2/8086-1}.
Data: Page 1, Address \& Bus: Page 2, Instructions: Page 26.
Intel Corporation, September 1990.

\bibitem{intel8088}
\textit{Intel 8088 8-BIT HMOS MICROPROCESSOR 8088/8088-2}.
Data: Page 1, Address \& Bus: Page 2, Instructions: Page 26.
Intel Corporation, August 1990.

\end{thebibliography}

%----------------------------------------------------------------------------------------

\end{document}
