%%%%%%%%%%%%%%%%%%%%%%%%%%%%%%%%%%%%%%%%%
% Journal Article
% LaTeX Template
% Version 1.4 (15/5/16)
%
% This template has been downloaded from:
% http://www.LaTeXTemplates.com
%
% Original author:
% Frits Wenneker (http://www.howtotex.com) with extensive modifications by
% Vel (vel@LaTeXTemplates.com)
%
% License:
% CC BY-NC-SA 3.0 (http://creativecommons.org/licenses/by-nc-sa/3.0/)
%
%%%%%%%%%%%%%%%%%%%%%%%%%%%%%%%%%%%%%%%%%

%----------------------------------------------------------------------------------------
%	PACKAGES AND OTHER DOCUMENT CONFIGURATIONS
%----------------------------------------------------------------------------------------

\documentclass[twoside,twocolumn]{article}

\usepackage{blindtext} % Package to generate dummy text throughout this template 

\usepackage[sc]{mathpazo} % Use the Palatino font
\usepackage[T1]{fontenc} % Use 8-bit encoding that has 256 glyphs
\linespread{1.05} % Line spacing - Palatino needs more space between lines
\usepackage{microtype} % Slightly tweak font spacing for aesthetics

\usepackage[english]{babel} % Language hyphenation and typographical rules

\usepackage[hmarginratio=1:1,top=32mm,columnsep=20pt]{geometry} % Document margins
\usepackage[hang, small,labelfont=bf,up,textfont=it,up]{caption} % Custom captions under/above floats in tables or figures
\usepackage{booktabs} % Horizontal rules in tables

\usepackage{lettrine} % The lettrine is the first enlarged letter at the beginning of the text

\usepackage{enumitem} % Customized lists
\setlist[itemize]{noitemsep} % Make itemize lists more compact

\usepackage{abstract} % Allows abstract customization
\renewcommand{\abstractnamefont}{\normalfont\bfseries} % Set the "Abstract" text to bold
\renewcommand{\abstracttextfont}{\normalfont\small\itshape} % Set the abstract itself to small italic text

\usepackage{titlesec} % Allows customization of titles
\renewcommand\thesection{\Roman{section}} % Roman numerals for the sections
\renewcommand\thesubsection{\roman{subsection}} % roman numerals for subsections
\titleformat{\section}[block]{\large\scshape\centering}{\thesection.}{1em}{} % Change the look of the section titles
\titleformat{\subsection}[block]{\large}{\thesubsection.}{1em}{} % Change the look of the section titles

\usepackage{fancyhdr} % Headers and footers
\pagestyle{fancy} % All pages have headers and footers
\fancyhead{} % Blank out the default header
\fancyfoot{} % Blank out the default footer
\fancyhead[C]{CPU Bitness $\bullet$ August 2020} % Custom header text
\fancyfoot[RO,LE]{\thepage} % Custom footer text

\usepackage{titling} % Customizing the title section

\usepackage{hyperref} % For hyperlinks in the PDF

\usepackage{pdflscape} % landscape orientation
\usepackage{afterpage} % allow moving sections to the next page
\usepackage{array} % allows specifying table column widths

%----------------------------------------------------------------------------------------
%	TITLE SECTION
%----------------------------------------------------------------------------------------

\setlength{\droptitle}{-4\baselineskip} % Move the title up

\pretitle{\begin{center}\Huge\bfseries} % Article title formatting
\posttitle{\end{center}} % Article title closing formatting
\title{CPU Bitness} % Article title
\author{%
\textsc{Simon Rupf} \\[1ex] % Your name
\normalsize \href{mailto:simon@rupf.net}{simon@rupf.net} % Your email address
%\and % Uncomment if 2 authors are required, duplicate these 4 lines if more
%\textsc{Jane Smith}\thanks{Corresponding author} \\[1ex] % Second author's name
%\normalsize University of Utah \\ % Second author's institution
%\normalsize \href{mailto:jane@smith.com}{jane@smith.com} % Second author's email address
}
\date{\today} % Leave empty to omit a date
\renewcommand{\maketitlehookd}{%
\begin{abstract}
\noindent When discussing properties of a computers processing power, one particular
property often used to classify a system is it's central processing unit's (CPU) bitness.
A CPU is said to be, for example, of the 8-Bit type and therefore compared to other
systems sporting CPUs of the same type. When looking at such a CPU more in-depth the
situation is of course more complex. This article tries to collect those details over
a broad range of CPUs, allowing for a more nuanced comparison of these CPUs. It also
attempts to highlight the somewhat arbitrary nature of these CPU Bitness categorizations.
\end{abstract}
}

%----------------------------------------------------------------------------------------

\begin{document}

% Print the title
\maketitle

%----------------------------------------------------------------------------------------
%	ARTICLE CONTENTS
%----------------------------------------------------------------------------------------

\section{Introduction}

\lettrine[nindent=0em,lines=3]{L} orem ipsum dolor sit amet, consectetur adipiscing elit.
\blindtext % Dummy text

\blindtext % Dummy text

%------------------------------------------------

\section{Methods}

Maecenas sed ultricies felis. Sed imperdiet dictum arcu a egestas. 
\begin{itemize}
\item Donec dolor arcu, rutrum id molestie in, viverra sed diam
\item Curabitur feugiat
\item turpis sed auctor facilisis
\item arcu eros accumsan lorem, at posuere mi diam sit amet tortor
\item Fusce fermentum, mi sit amet euismod rutrum
\item sem lorem molestie diam, iaculis aliquet sapien tortor non nisi
\item Pellentesque bibendum pretium aliquet
\end{itemize}
\blindtext % Dummy text

Text requiring further explanation\footnote{Example footnote}.

%------------------------------------------------

\section{Results}

\afterpage{
\clearpage % Flush earlier floats (otherwise order might not be correct)
\thispagestyle{empty} % empty page style (?)
\begin{landscape} % Landscape page
\begin{table}
\centering % Center table
\begin{tabular}{lrrrrrm{2.5cm}m{6.5cm}}
\toprule
&&\multicolumn{4}{c}{Width in Bits} \\
\cmidrule(r){3-6}
CPU & Year & Instructions & Data & Address & Bus & Used in & Notes \\
\toprule
Intel 8008 & \cite{intelchips} 1972 & \cite[p.3]{intel8008} 8/16/22 & \cite[p.3]{intel8008} 8 & \cite[p.1]{intel8008} 14 & \cite[p.2]{intel8008} 8 & MCS-8, Mark-8 & bi-directional bus \\
\hline
Intel 8080 & \cite{intelchips} 1974 & \cite[p.8]{intel8080} 8/16/24 & \cite[p.1]{intel8080} 8 & \cite[p.1]{intel8080} 16 & \cite[p.2]{intel8080} 24 & Altair 8800 & separate address and data bus \\
\hline
Intel 8088 & 1978 & 8-48 & 8/16 & 16/20 & 20 & IBM PC & data is fetched 8 bits at a time, but internally stored in 16 bit registers \\
\hline
Intel 8086 & \cite{intelchips} 1978 & 8-48 & 16 & 16/20 & 20 & IBM PS/2 & access to addresses beyond 16 bits requires banking of segments \\
\hline
Zilog Z80 & 197x & 8-24 & 8/16 & 16 & 8 & Sinclair ZX Spectrum & designed to be binary compatible with Intel 8080 \\
\hline
MOS 6502 & 197x & 8-24 & 8 & 16 & 8 & Commodore 64 & unlicensed clone of the Motorola 6800 \\
\hline
Motorola 6800 & 197x & 8-24 & 8 & 16 & 8 &  & very rarely used \\
\bottomrule
\end{tabular}
\end{table}
\captionof{table}{CPU Bitness table} % Add 'table' caption
\end{landscape}
\clearpage % Flush page
}

\blindtext % Dummy text

\begin{equation}
\label{eq:emc}
e = mc^2
\end{equation}

\blindtext % Dummy text

%------------------------------------------------

\section{Discussion}

\subsection{Subsection One}

A statement requiring citation \cite{Figueredo:2009dg}.
\blindtext % Dummy text

\subsection{Subsection Two}

\blindtext % Dummy text

%----------------------------------------------------------------------------------------
%	REFERENCE LIST
%----------------------------------------------------------------------------------------

\begin{thebibliography}{99} % Bibliography - this is intentionally simple in this template

\bibitem{intelchips}
\textit{Intel Chips timeline}.
Intel Corporation, 2012.

\bibitem{intel8008}
\textit{Intel 8008/8008-1 EIGHT-BIT MICROPROCESSOR}.
Intel Corporation, 1978.

\bibitem{intel8080}
\textit{Intel 8080A/8080A-1/8080A-2 8-BIT N-CHANNEL MICROPROCESSOR}.
Intel Corporation, 1986.

\bibitem{intel8080users}
\textit{Intel Intel 8080 Microcomputer Systems Users's Manual}, CHAPTER 2 THE 8080 CENTRAL PROCESSOR UNIT, p.15-34.
Intel Corporation, September 1975.

\end{thebibliography}

%----------------------------------------------------------------------------------------

\end{document}
